An investigation into Networking and Occam-\/\+Pi for Raspberry-\/\+Pi processors.

Search engines, social networks and scientific measurements are few examples of nowadays fields heavily relying on computational power. Parallel programming has been developed to make the most out of multi-\/core architectures. Yet, there are always underlying communications at some point between processes, applications, machines, clusters and so on. This honours project focussed on networking communications for parallel programming language occam. Robust and inherently parallel, it has been amongst the firsts to approach the parallel programming world 30 years ago. Evidently, occam has been provided a network architecture since then\+: in 2006, called p\+Ony. Nevertheless, in the last decade newer ones have emerged; being the starting point of this honours project charged of finding out a better network architecture and implementing it. It implicitly launched a “\+Back to occam” conquest where the ultimate aim would be to integrate this new architecture into occam. This was where net2 came into consideration, both designed and implemented by Doctor Kevin Chalmers, it demonstrated its efficiency and became number one contender for occam. However, occam is not of easy access\+: it is compiled with K\+Ro\+C then interpreted by the Transterpreter. Therefore, the candidate network architecture will have to go through these two layers. This projects intended to kick it off by developing an implementation in C language as both layers are so; its potential integration would then be much easier. As a final output of this honours project, a functional C implementation has effectively been developed and experimented over several Raspberry-\/\+Pis using a distributed application. 